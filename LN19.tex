% Created 2022-03-03 Thu 15:11
% Intended LaTeX compiler: pdflatex
\documentclass[11pt]{article}
\usepackage[utf8]{inputenc}
\usepackage[T1]{fontenc}
\usepackage{graphicx}
\usepackage{longtable}
\usepackage{wrapfig}
\usepackage{rotating}
\usepackage[normalem]{ulem}
\usepackage{amsmath}
\usepackage{amssymb}
\usepackage{capt-of}
\usepackage{hyperref}
\usepackage{fancyhdr}
\usepackage{amsmath}
\newcommand{\vect}[1]{\ensuremath{\mathbf{#1}}}
\newcommand{\cf}[2]{\ensuremath{\cfrac{#1}{#2}}}
\newcommand{\la}{\ensuremath{\langle}}
\newcommand{\ra}{\ensuremath{\rangle}}
\newcommand{\Ra}{\ensuremath{\Rightarrow}}
\newcommand{\La}{\ensuremath{\Leftarrow}}
\newcommand{\bse}{\ensuremath{\subseteq}}
\newcommand{\pse}{\ensuremath{\supseteq}}
\newcommand{\bs}{\ensuremath{\subset}}
\newcommand{\ps}{\ensuremath{\supset}}
\newcommand{\ZZ}{\ensuremath{\mathbb{Z}}}
\author{Fanping}
\date{\today}
\title{}
\hypersetup{
 pdfauthor={Fanping},
 pdftitle={},
 pdfkeywords={},
 pdfsubject={},
 pdfcreator={Emacs 26.3 (Org mode 9.5.1)},
 pdflang={English}}
\begin{document}

\tableofcontents

\setcounter{secnumdepth}{-1}
\section{LN 19}
\label{sec:org979f7b4}
\subsection{Prime Ideal}
\label{sec:org942a1d9}
\begin{itemize}
\item Definition: Prime Ideal
An ideal \(I \subseteq R\), \(R\) is a ring is called prime
if whenever \(a, b \in I\) then \(a \in I\) or \(b \in I \,\forall a, b \in R\).

\item Proposition
For a commutative ring \(R\), \(I \subseteq R\) is a prime ideal iff \(\cfrac{R}{I}\)
is an integral domain (no-zero divisors)
\begin{itemize}
\item Proof
\begin{itemize}
\item \(\Rightarrow\)
\end{itemize}
Suppose that \((a + I) (b + I) = 0\) ie \(ab \in I\) but \(I\) is prime
So \(a \in I\) or \(b \in I\) which means \(a + I = 0\) or \(b + I = 0\).

\begin{itemize}
\item \(\Leftarrow\) if \(ab \in I\) then \((a + I)(b + I) = 0\) in $\cfrac{R}{I}$
\end{itemize}
TODO\textsubscript{sc1}
\end{itemize}

\item Example
\(\mathbb{Z}\) as a ring - it has ideals \(n\mathbb{Z}\) for \(n \in \mathbb{Z}\)
which ones are prime?

\(p\mathbb{Z}\) for \(p\) a prime are the prime ideals.

What is \(\cfrac{\mathbb{Z}}{p\mathbb{Z}}\)?
By the ``Proposition'', this is an integral domain. This is a field!

\item Fact
Any finite itegral domain is a field.
\begin{itemize}
\item Proof (By ``Pigeon Hole Principle'')
Suppose that \(R\) is a finite integral domain and \(a \in R, a \neq 0\).
We need to find \(b\) such that \(ab = 1\).

\begin{itemize}
\item Define: \(F : R \to R, r \mapsto ar\).
\item Claim: \(F\) is a \(1-1\).

Suppose \(ar = as\) for some \(r, s \in R\) then \(a(r - s) = 0\)

But \(R\) is an integral domain. So either \(a = 0\) or \$r - s = \$.

But \(a \neq 0\) so \(r = s\)

\(R\) is finite so \(F\) is onto!
\end{itemize}

So for some \(B \in R, ab = 1\).

So \(R\) is a field!
\end{itemize}
\end{itemize}
\subsection{Maximal Ideal}
\label{sec:orgf60c60c}
\begin{itemize}
\item Definition: Maximal ideal
We call an ideal \(I \subseteq R\) maximal if \(I \neq R\) but if \(I \subseteq J \subseteq R\),
J is an ideal, then \(I = J\) or \(J = R\)

\item Proposition:
For a commutative ring with \(1\), \(I\) is maximal iff \(\cfrac{R}{I}\) is a field.
\begin{itemize}
\item Proof
\begin{itemize}
\item \(\La\)
The only ideals in a field \(F\) are \(F\) and \(\{0\}\).
Because non-zero elements have multi-inverse
TODO
If a TODO

\item \(\Ra\)
Suppose that \(a + I \neq 0\) ie \(a \notin I\) (\(I\) is maximal).
Genereate the smallest ideal containing \(a\) and \(I\).

\(\la a , I\ra = \{ ra + rb: r \in R, b \in I \} \subset I\).

So \(R = \la a , I\ra\) so \(1 = ra + rb\) for some \(r \in R, b \in I\)

This means \((r + I)(a + I) = 1 + I\) in \(\cfrac{R}{I}\)

Si \(a + I\) has multiplicative inverse and \(\cfrac{R}{I}\) is a field.
\end{itemize}
\end{itemize}

\item Carollary:
If \(R\) is a commutative ring with \(1\) then any maximal ideal is prime
\begin{itemize}
\item Proof
If \(I\) is maximal then \(\cf{R}{I}\) is a field hense an integral domain

So \(I\) is prime.
\end{itemize}

\item Example
\(\cf{\ZZ}{p\ZZ}\) is both a field and an integral domain for \(p\) a prime
so \(p\ZZ\) is both prime and maximal in \(\ZZ\).
\end{itemize}
\subsection{PID: principlae ideal domain}
\label{sec:orgf5c546c}
Notice in \(\ZZ\), every idael is 1-generated - of the form \(\la n \ra\) where \(n \in \ZZ\).
$n =$ the smallest ideal TODO n, ie \(n\ZZ\).

\begin{itemize}
\item Definition: PID
\end{itemize}
A integral domain \(M\) where all dieals are 1-generated is called a principle ideal domain. (PID).

TODO
\begin{itemize}
\item Example
TODO
\item Division algorithm
If \(f, g \in F[x]\) then there are unique \(q,r \in F[x]\) such that
\(g = fq + r\) with \(degree(r) < degree(f)\).

\begin{itemize}
\item Long division algorithm
\(f = a_nx^n + ... + a_0\)
\(b = b_nx^m + ... + b_0\)

TODO\textsubscript{sc3}

\item Ideals in \(F[x]\)
Suppose \(I \bse F[x], I \neq \{0\}\)

What does \(I\) look like?

\(I = \la f\ra\) the ideal generated by \(f\).

where \(f\) has the least degree among elements of \(I\).

\begin{itemize}
\item Why?
Suppose \(g \in I\).
Use the division algorithm to divide \(g\) by \(f\), ie $g = fq + r $
with \(degree(r) < degree(f)\)

But notice \(g \in I, fq \in I\) so \(r\) must be in \(I\).
So \(r = 0\)

Which means, \(g = fq\). So \(g \in \la f\ra\)
\end{itemize}
\end{itemize}
\end{itemize}


\end{document}