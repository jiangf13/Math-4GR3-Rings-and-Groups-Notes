% Created 2022-03-02 Wed 14:23
% Intended LaTeX compiler: pdflatex
\documentclass[11pt]{article}
\usepackage[utf8]{inputenc}
\usepackage[T1]{fontenc}
\usepackage{graphicx}
\usepackage{longtable}
\usepackage{wrapfig}
\usepackage{rotating}
\usepackage[normalem]{ulem}
\usepackage{amsmath}
\usepackage{amssymb}
\usepackage{capt-of}
\usepackage{hyperref}
\author{Fanping}
\date{\today}
\title{}
\hypersetup{
 pdfauthor={Fanping},
 pdftitle={},
 pdfkeywords={},
 pdfsubject={},
 pdfcreator={Emacs 26.3 (Org mode 9.5.1)},
 pdflang={English}}
\begin{document}

\tableofcontents

\section{LN 18}
\label{sec:org1e5cb7a}
Groups. \(M\) a math structure \(\implies\ Aut(M)\) - auto of \(M\)
\subsection{Where do rings show up?}
\label{sec:orga7996c3}
The why of rings
\subsubsection{Exmple: \((A, +)\) and abelian group}
\label{sec:org0c95ee7}
\(End(A) = \{f : F : A \to A \}\)
\(End(A)\) has 2 natural binary operations:
\begin{itemize}
\item \(+\)
\(f, g \in\ End(A)\) TODO
\item \(*\)
\(f, g \in\ End(A)\) then \(f \cdot g\ := f \circ g\)
This is still a homomorphism. \((End(A), \cdot)\) is associative and \(id_A\)
is the identity.
\item Check distributivity!
\((End(A), +, \cdot)\) is a ring.
\end{itemize}

\subsubsection{Example: What is we look at \(End(\mathbb{Z}, +)\)?}
\label{sec:org4de7b1a}
All \(f : \mathbb{Z} \to \mathbb{Z}\) and determined by where \(1\) goes.
Say \(f(1) = n\) and \(g(1) = m\), what is \(g(f(1))\)?

So \(End(\mathbb{Z}, +) = (\mathbb{Z}, + \cdot)\).

\subsubsection{Example: \(V = \mathbb{C}^n\)}
\label{sec:org1731a31}
\(V\) is complex n-space and look \(\mathbb{R}\), the set of all linear
tranformations \(F: V \to V\)
\begin{itemize}
\item \(+\) on \(\mathbb{R}\)
\(f, g : V \to V\) on linear transformations, so is \(f + g\)
\item \(\cdot\) on \(\mathbb{R}\)
is just composition \(f \cdot g = f \circ g\).
\end{itemize}

Check that \((R, +, \cdot)\) is a ring with 1 - \(1 = id_V\)

This is \(M_n(\mathbb{C})\) - \(n \times n\) complex matrices.

\subsubsection{Subring}
\label{sec:org5233038}
If \(R\) is a ring and \(S \subseteq R\), then we say \(S\) is a subring of \(R\)
if \(S\) with \(+\) and \(\cdot\) from \(R\) restricted to S is a ring

\begin{itemize}
\item Example \((\mathbb{Z}, +, \cdot)\)
Then $n\mathbb{Z} = \{nm : m \in \mathbb{Z}\} \subseteq \mathbb{Z}$is a subring
but noice if \(n \neq \pm 1\) then \(n\mathbb{Z}\) does not have a 1.
\end{itemize}

If \(R\), \(S\) TODO

\subsubsection{Ring homomorphism}
\label{sec:orgaa1a9af}
If \$\(\phi\) : R \(\to\) S\$is a ring homomorphism then \(ker(\phi) = \{a \in R : \phi(a) = 0\}\)

Fact: \$ker(R)\$is closed under \(+\)
and if \(a \in\ ker(\phi),\ r \in R\) then \(ar, ra \in\ ker(\phi)\)

\subsection{Ideals}
\label{sec:orgdfd29eb}
If \(R\) is a ring and \(I \subseteq R\) then \(I\) is an ideal
if \((I, +)\) is an abelian group and \(\forall a \in I, r \in R, ar, ra \in I\)

Notice TODO

\begin{itemize}
\item Example: What are ideals of \((\mathbb{Z}, +, \cdot)\)?
\(\{0\}\), \(p\mathbb{Z}\),
\(n\mathbb{Z}\ \forall n \in \mathbb{Z}\) (this is general ideal in $\mathbb{Z}$)
\end{itemize}

\subsubsection{Why do we care about ideals?}
\label{sec:org61f8ffd}
We want to make sense of quotients!

Given \(BI \subseteq R\), \(R\) some ring and \(I\) an ideal. we want to define \(\cfrac{R}{I}\) is a ring.

TODO


\subsubsection{How do we define \((r + I)(s + I)\) for \(r, s \in R\)?}
\label{sec:orgb211b2d}
Define \((r + I)(s + I) = rs + I\). we need to se that this is well-defined.
If \(r + I = r' + I\), ie \(r - r' \in I\) then \((r - r')s \in I\) because I is an ideal.
\(\because I\)  is an ideal, so \(rs + I = r's + I\).

We do something similar for \(s\) so  multiplication is well-defined.

\begin{itemize}
\item Check: $\cfrac{R}{I}$is a ring with these operations.
\begin{itemize}
\item Example \(\mathbb{Z}\) has \(n\mathbb{Z}\) as ideals for \(n \in \mathbb{Z}\)
What is \(\cfrac{\mathbb{Z}}{n\mathbb{Z}}\) as a ring ?

This is arithmatic mod \(n\).

\textbf{1st Isomorphism Theorem}: If \(\phi: R \to S\) is a ring homomorphism
then \(\cfrac{R}{ker(\phi)} \cong Im(\phi) \subseteq S\).

Notice: Every ideal is th kernel of some ring homomorphisms.
\(\phi: R \to \cfrac{R}{I}, i \mapsto r + I\), then \(ker(\phi) = I\).
\end{itemize}
\end{itemize}

TODO: hard to draw the diagram
Rings \(\to +, \cdot\), satisfying \ldots{}
Rings with identity                                   Rings without identity.
Commutative                         Non-commutative
Integral domain                              Division ring (multi-inverse of non-zero elements
(No 0 divisors                               but not nes. commutative)
ie if \(ab= 0\) then \(a = 0\) or \(b = 0\))
Fields: division ring and integral domain
(every non-zero elements has a multiplicative inverse)

Judson, pg 192 quaternions TODO
\end{document}