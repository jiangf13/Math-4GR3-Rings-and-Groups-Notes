% Created 2022-02-28 Mon 14:35
% Intended LaTeX compiler: pdflatex
\documentclass[11pt]{article}
\usepackage[utf8]{inputenc}
\usepackage[T1]{fontenc}
\usepackage{graphicx}
\usepackage{longtable}
\usepackage{wrapfig}
\usepackage{rotating}
\usepackage[normalem]{ulem}
\usepackage{amsmath}
\usepackage{amssymb}
\usepackage{capt-of}
\usepackage{hyperref}
\author{Fanping}
\date{\today}
\title{}
\hypersetup{
 pdfauthor={Fanping},
 pdftitle={},
 pdfkeywords={},
 pdfsubject={},
 pdfcreator={Emacs 26.3 (Org mode 9.5.1)},
 pdflang={English}}
\begin{document}

\tableofcontents

\section{Lecture Note 17}
\label{sec:orgfa5b31e}
\subsection{how to classify groups with semi-directed products}
\label{sec:orgf735c4d}
\subsubsection{Warm up exercise: \(|G| = 6\)}
\label{sec:orgcf3e942}
Say \(G_{13}\) of size \(6\). Let \(N \lhd G, |N| = 3\)
Let \(H\) be a group, \(|H| = 2, H \subset G\).
Notice \(H \cap N = \{e\}\).
So \(G = NH\).

What is \(Aut<N>\)? automorphism group. \(N \cong C_3\).
\(Aut<N> \cong\)

TODO

There are 2 homomorphisms \(\phi: H \to Aut<N>\), \(x \to id_N\), these are \(\phi_1\).
and the other one \(\phi_2\)

\(C_3 \times_{\phi_1} C_2\), muliplication in this exmaple is
\(<n_1, h_1> <n_2, h_2> = n_1 \phi_1(h_1, n_2) h_1h_2 = n_1n_2 \cdot h_1h_2\)
ie \(C_3 \times C_2\) or \(C_6\)

\(C_3 \times_{\phi_2} C_2 \cong S_3\)
\subsubsection{Classify all groups \(|G| = 12\)}
\label{sec:org23aa595}
\(12 = 2^2 \cdot 3\), so we have 2-Sylow subgroup of size \(4\) and a 3-Sylow
subgroup of size \(3\).
\begin{enumerate}
\item Example
\label{sec:orgbd7fa42}
\(C_12 \cong C_4 \times C_3\), \(C_2 \times C_2 \times C_3\),
\(A_4\)
Dihearal group of with \(12\) elements \(\to D_6\).

What about \(S_3 \times C_2\)? Is it actually \(S_3 \times C_2 \cong D6\)?

There are \(3\) non-isomorphic non-abelian group of order \(12\)
but the missing one is called the dicyclic group of order \(12\).

\(G\) of size \(12\) and \(n_3 \equiv 1\ mod\ 3\) and \(n_3 | 4\).
so either \(n_3 = 1\) or \(4\). If \(n_3 = 1\) then the 3-Sylow subgroup is
normal otherwise we have \(n_3 = 4\).

If we have \(4\) 3-Sylow subgroups, then \(8\) elements of G have order \(3\)
and we also have the identity \(60\) this only allows \(3\) other elements so
\(n_2 = 1\). In this case the 2-Sylow subgroups is normal.

\begin{itemize}
\item Case 1
Let \(n_2 = 1\) with 2-Sylow subgroup \(N\) and some 3-Sylow subgroup \(H\).
\(N \cap H = \{e\}\) and \(G = NH\). So \(G\) is a semi-direct product. (Hint: We
only get 1 non-abelian up to isomorphism.)

\item Case 2
\(n_3 = 1\) so we have normal 3-Sylow subgroup \(N\) and some 2-Sylow
subgroup \(H\). Again \(G = NH\) and \(G\) is a semi-direct product. There are \(2\)
non-abelian examples here up to isomorphism.
\end{itemize}

Notice if \(G \cong N \times H\) then \(G\) is abelian.
\end{enumerate}

\subsection{Part 2 of Math 4GR3: Rings and Fields}
\label{sec:org9cce1d2}
\subsubsection{What is a Ring?}
\label{sec:orgb988ce3}
A set \(R\) with \(2\) binary operations \(+\) and \(\cdot\) is called a ring if
\begin{itemize}
\item \((R, +)\) is an abelian group.
Usually write \(0\) for the identity

\item \(\cdot\) is associative
\(x \cdot(y \cdot z) = (x \cdot y) \cdot z \forall x,y,z \in R\)

If there is a unit, we usually write it as \(1\), i.e.
\(1 \cdot x = x \cdot 1 = x \forall x \in R\).

\item Distributivity
\(x \cdot (y + z) = x \cdot y + x \cdot z\) (left-distributivity)
\((y + z) \cdot x = y \cdot x + z \cdot x\) (right-distributivity)
\(\forall x,y,z \in R\).
\end{itemize}

\begin{enumerate}
\item Example
\label{sec:org762815e}
\begin{itemize}
\item \(Z\) iwth usual \(+\) and \(\cdot\) is a ring
\item \(\mathbb{Q}, \mathbb{R}, \mathbb{C}\) with usual \(+\) and \(\cdot\) are rings.
In fact, every non-zero element has a multiplicative inverse, so there are fields.
(Multiplication is commutative as well).
\item If we have multiplicative inverse but not commutative TODO
\item Polynomial rings: \(\mathbb{Z}\{x\}, \mathbb{R}\{x\}, \mathbb{C}\{x\}\)
with usual \(+\) and \(\cdot\)
\item Non-commutative example of a ring:
\(M_n(\mathbb{R})\) or \(M_n(\mathbb{C})\), matrix rings (\(n \times n\) with
entries in \(\mathbb{R}\) or \(\mathbb{C}\))
\item Naturual example of a ring without \(1\)
\(R = \{F : \mathbb{R} \to \mathbb{R}\}\), such that \(\displaystyle \int_{-\infty}^{+\infty} |f|\,dx < \infty\)
with usual addition of functions and usual multiplication. But natural
choice for \(1\) would be the constant function \(1\) but \(1 \notin \mathbb{R}\)
\end{itemize}
\end{enumerate}
\end{document}